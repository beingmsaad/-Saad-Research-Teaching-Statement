\documentclass{NSF}

\usepackage{xspace}
\newcommand{\BfPara}[1]{{\noindent\textbf{#1.}}\xspace}
\newcommand{\etal}{{\em et al.}\xspace}
\newcommand{\etc}{{etc.}\xspace}
\newcommand{\ie}{{\em i.e.}\xspace}
\newcommand{\eg}{{\em e.g.,}\xspace}
\usepackage{marvosym}
% \usepackage{libertine}
% \usepackage{libertinust1math}
% \usepackage[T1]{fontenc}
\usepackage[bitstream-charter]{mathdesign}
\usepackage[T1]{fontenc}
\begin{document}

\title{Teaching Statement -- \name{Muhammad Saad }}
\thispagestyle{empty}
Teaching is an integral part of academia since it plays a critical role in shaping our future. In the applied field of computer science, teaching requires effective knowledge communication, practical skill development, and inspirational mentorship. Moreover, teaching offers learning for both instructors and students. By joining academia, I will leverage the teaching opportunities to mentor young students and enable them to fully harness their potential and creative pursuits.  


\subsection*{Teaching Philosophy and Style}
My core teaching philosophy pivots on the principle of {\em effective learning} which requires embracing teaching as a sanctimonious profession rather than an occupation. In my view, {\em effective learning} includes interactive classroom environment, maximum resource utilization, constructive feedback, two-way learning, and maintaining academic integrity. In the following, I will discuss my teaching philosophy and style based on these key variables.   


\vspace{1mm}
\BfPara{Interactive Classroom Environment} In my experience, an interactive classroom environment significantly increases the learning potential of students. Generally, a classroom is a diverse environment and students have different aptitudes and learning curves. This can lead to communication barriers among students which can be addressed through an interactive environment. An interactive environment ensures that (1) students gain confidence to express their thoughts to the instructor and their colleagues, (2) an instructor can comfortably engage with each student and help them when needed, and (3) students can execute responsible teamwork during coursework and projects. To provision such an environment, some useful methodologies can be adopted including (1) addressing students by their names and preferred pronouns, (2) promoting group activities and presentations, (3) encouraging peer feedback, and (4) adopting a friendly classroom environment. Healthy interactions remove the communication barriers and promote collective learning in the classroom. Therefore, the foundation of my teaching philosophy is in a healthy, productive, and interactive classroom environment. 

\vspace{1mm}
\BfPara{Resource Utilization} For effective knowledge communication, an instructor should maximally utilize all the available resources. Resource utilization include using(1) multi-media tools (\ie PowerPoint presentations), (2) classroom board for problem solving, and (3) lecture handouts before or after the lecture. PowerPoint presentations are helpful in giving a high level idea about the topic, and simplify complex workflows through animations. The classroom board can be used to demonstrate the problem solving methodology or elaborating a core concept in detail. During my studies, I always replicated the instructor's problem solving approach in my exams and found it to be helpful. Lastly, lecture handouts are tools of reinforcement that can maximize the classroom learning. Students can glance at the lecture handouts during spare time or before an exam to revise and assess their preparations. The COVID-19 pandemic has changed the way we use conventional resources for lectures. Remote learning procedures involve managing new tools that can closely emulate the in-person classroom environment. For such situations in the future, I will ensure that I have the right set of tools that do not affect student learning provide them an equivalent in-person classroom experience. 

\vspace{1mm}
\BfPara{Incorporating Feedback} In my view, a major component of {\em effective learning} is being reachable to students and providing them timely feedback. Instructors typically have dedicated office hours for students outside the classroom. However, sometimes students do not make use of the office hours due to other commitments. During the exam seasons, students typically experience more issues which cannot be addressed in the limited office hours. As an instructor, I will ensure that students are given extra attention during the exam period, even outside the office hours. I will schedule extra coaching classes around the exam season if needed. Along with provisioning accessibility to students, it is also important to give them timely feedback regarding their performance and progress. Timely feedback helps in troubleshooting difficulties experienced by students. Moreover, since students can have varying aptitude, it also helps in identifying students who need extra counseling regarding certain topics. The best way to ensure a timely feedback is to grade quizzes and assignments on time and giving detailed remarks on assignments. 


\vspace{1mm}
\BfPara{Two-way Learning} Typically as students, we perceive that classroom learning experience is one-way, and students benefit from the instructors' knowledge. However, in reality, academics we are always learning irrespective of their role and position. As instructors, we learn from (1) valuable questions raised in the classroom, (2) new advancements in the domain being taught, and (3) feedback on the teaching style. Instructors who work hard on their lectures and try to improve their knowledge, inspire their students to follow them. As an instructor, I will earnestly prepare my course lectures, continuously upgrade my knowledge of the domain, and try learn from my students. 

\vspace{1mm}
\BfPara{Academic Integrity} The final component of my teaching philosophy is to uphold and enforce the academic values. As I mentioned earlier, teaching must be considered a sanctimonious profession since instructors have the position to influence young minds. Therefore, it is important that instructors also pass on the academic values to their students. At UCF, we have a zero tolerance policy towards plagiarism, cheating, and violations of academic integrity. As an instructor, I will uphold those values and clearly communicate my expectation to students regarding responsible conduct. At the same time, I earnestly believe in leading by example rather than leading by force. If students are facilitated to the maximum level, it is less likely that they resolve to unethical practices. Leading by example promotes a positive academic culture which is desirable for the community at large. 




\subsection*{Teaching Experiences}
In the last five years, I had the privilege of teaching and training students ranging from the high school to Ph.D programs. I have also served as a teaching assistant for two courses, namely Foundations of Computer Security and Privacy, and Natural Language Processing. Alongside, I have informally mentored undergraduate students in their final year projects and helped them pursue their graduate studies. The diversity of experience gained from interacting with students at various levels helped me understand the requirements of constructive teaching at each level. Below, I succinctly describe my teaching experience over the last five years.

During my Masters degree at LUMS, my advisor instructed me to help the undergraduate students in their final year projects. For that purpose, I conducted weekly meetings with those students and assigned them project tasks. Based on the project requirements, I helped them select the relevant coursework \ie Machine Learning and Data Mining. Accordingly, I also attended those courses to stay updated with the course material and discuss them during weekly meetings. Among the four students I assisted, three pursued their graduate studies in the USA.

At UCF, I volunteered for the outreach program organized by Camp Connect. In that activity, I taught the fundamentals of computer security to high school students. I observed that students in the younger generation have a higher technology awareness than our generation. On one hand, this made it easier for me to communicate with them, on the other hand, it also made it challenging to stay updated with new applications they use in their daily lives. To make the best use of the situation, I structured more interactive lectures that enabled a collaborative learning experience for me and the students. In the coming years, high school students will collaborate with undergraduates and Ph.D. students. NSF has a Computing K-12 Education plan that is focused on involving high school students in computer science education. I am hoping that my experience will be useful for such programs in the future. 


As at teaching assistant at UCF, I had a profound experience with students that motivated me to join academia. I assisted Professor David Mohaisen in Foundations of Computer Security and Privacy, and Natural Language Processing. My responsibilities included helping Professor Mohaisen in preparing the course material, grading the assignments, and conducting the exams. For each assignment, I gave a detailed and comprehensive feedback to each student to help them reflect on their shortcomings. I also held office hours to help students with their assignment questions and other course related queries. Following the teaching philosophy described earlier, I ensured that I was accessible and responsive during the entire semester. For the security course exams, Professor Mohaisen and I took a unique approach to prepare exams based on real-world parallels. The exam questions were designed using interesting case studies each of which captured a security model taught in the class. We observed that students performed well in that exam, demonstrating the fact that interest can stimulate analytical ability. Moreover, for both courses, we assigned course projects to the students and encouraged them select projects that were closely related to the course material as well as their Ph.D. To further sharpen their skills, the project report and deliverables were modeled on the research paper format (\ie IEEE style report formatting and code repositories). For students who were not familiar with the format, I provided detailed instructions and project samples as guidelines. The intent of such formatting was to help students in their coursework as well as Ph.D. In the future, I will apply the teaching experience learned from Professor Mohaisen to the courses that I teach as a faculty member. 


\vspace{2mm}
\BfPara{Teaching Interest} During my coursework and research, I have covered a wide range of topics in computer science and mathematics including computer and network security, malware analysis, design and analysis of algorithms, computer networks, operating systems, advanced computer architecture, machine learning, and advanced applied stochastic processes. If required, I can teach similar courses at undergraduate and graduate levels. I will be particularly interested in teaching courses related to security and networks since they are rapidly evolving domains with various research challenges. Based on research experience, I will revise the course content to include new concepts and research directions for students, and help them in solving cutting edge research problems. 

Additionally, I will introduce new courses related to blockchains, smart contracts, social networks, and Internet measurements. These courses have been introduced in notable universities to familiarize students with emerging concepts in computer science and train them for the job market. I am also interested in teaching the special topics courses where students learn to to read, summarize, and critique other research papers. Special topics help students in selecting their Ph.D. research topics research topics.


\vspace{2mm}
\BfPara{Student Advising} I was fortunate to be advised by the most skillful advisors during my Masters and Ph.D. When I become an advisor, I will follow the same advising style that has helped my and my colleagues during our Ph.D. My own learning is that advising is the skill to train students, help them achieve their milestones, and enable them to conduct independent research. As an advisor, I will initially help my students to understand the research culture and select the topic that they want to work on. Concurrently, I will introduce them to the preliminaries of paper writing, paper reviewing, rebuttals, peer review process, and academic milestones. Once the students acclimatize with the research culture and develop preliminary skills, I will help them identify the core research problems in the areas that they are interested in. The advising style will be hands on training the first year where students will learn the basics of problem solving and effectively translating their coursework to support their research. Eventually, once the students start to publish, they will independently identify new research problems in their domain. To steer the research in my group, I will regularly meet with students on weekly basis. I will also ensure a healthy and interactive lab environment where students can peacefully work and engage with their colleagues. Additionally, I will ensure that my students strike a work life balance and do not work on the weekends. Another important aspect of research advising is helping students deal with paper rejections. Paper rejections can be stressful and discouraging for students, especially in the start of their Ph.D. To prevent that, I will council them on handling rejections and improving their works through iterative feedback. 


I hope that my teaching philosophy and advising style proves to be effective and helpful for my students. Teaching is a great opportunity to bring positive change in the society by communicating knowledge and ethical values to the younger generation. As teachers, we can quickly get the feedback of our endeavors through the performance of our students. I am therefore looking forward to the teaching opportunities in the future.  

\end{document}


%Foundations of Computer Security and Privacy 
%Natural Language Processing 