\documentclass{NSF}

\usepackage{xspace}
\newcommand{\BfPara}[1]{{\noindent\textbf{#1.}}\xspace}
\newcommand{\etal}{{\em et al.}\xspace}
\newcommand{\etc}{{etc.}\xspace}
\newcommand{\ie}{{\em i.e.}\xspace}
\newcommand{\eg}{{\em e.g.,}\xspace}
\usepackage{marvosym}
\usepackage{libertine}
\usepackage{libertinust1math}
\usepackage[T1]{fontenc}
\begin{document}

\title{Teaching Statement -- \name{Muhammad Saad }}
\thispagestyle{empty}
Teaching is an integral part of academia and it plays a critical role in shaping our future. In the applied field of computer science, teaching requires effective knowledge communication, practical skill development, and inspirational mentorship.  Constructive teaching is a powerful learning experience, both for teachers and students. My key motivation to join academia is to leverage the teaching opportunities and impart skills, values, and experience to young minds and enable them to fully harness their potential and creative pursuits.  


\subsection*{Teaching Philosophy and Style}
My core teaching philosophy pivots on the principle of {\em effective learning}, which requires acknowledging teaching as a sanctimonious profession rather than an occupation. I believe that {\em effective learning} is a multi-variate approach towards minimizing the gap between knowledge to be imparted and the understanding of students. In the following, I briefly explain the key constructs of {\em effective learning}.

The most important aspect in knowledge communication is the classroom environment. From experience, I have learned that the most productive classroom environment is an interactive environment which promotes collaborative education. Since each student has a unique aptitude and personality, therefore, a classroom is a diverse environment with unique challenges. An interactive environment ensures that (1) students gain confidence to express their thoughts to the instructor and colleagues, (2) an instructor is comfortable to engage with each student and help them where needed, (3) problem solving and learning are collective and mutual, and (4) an innate sense of responsible teamwork develops among students regarding their coursework and projects. To enable such an interactive environment, there are useful methodologies that can be adopted including (1) getting to know students and addressing them by their names and preferred pronouns, (2) promoting group activities and presentations, (3) encouraging peer feedback, and (4) adopting a friendly classroom environment. Therefore, the foundation of my teaching philosophy is in a healthy, productive, and interactive classroom environment. 

The second key component of {\em effective learning} is the ability to utilize available resources in order to maximize the knowledge communication. This has become a major challenge especially after COVID-19 when we had to adapt remote learning and online lectures. From my experience, I have realized that a good teaching style involves a good mix of using multimedia tools such as PowerPoint presentations, classroom board for problem solving, and handouts before and after the lecture. PowerPoint presentations are helpful in giving a high level idea about the topic and provide an intuitive understanding of complicated workflows. A board can be used to breakdown a key concept and demonstrate the methodology of problem solving. I have often used instructor's problem solving approach on paper during the exams, and found it to be useful. Lastly, a handout can be a reinforcement tool that summarizes the core elements of a lecture. Students can often take a glance at the handout during spare time or before an exam to revise and assess their preparations. The COVID-19 pandemic has pushed us to adapt to new methodologies of remote learning. It is therefore important that instructors know the right tools that can closely emulate the in-person classroom environment. For such situations, I will ensure that I have the right tools that substitute the classroom board and I can effectively engage with students when they have questions. 


The third component of {\em effective learning} is being reachable and managing feedback. Instructors typically have dedicated office hours to help students in their work. However, as in most academic institutions, students typically experience more issues during the exam period. As an instructor, I will ensure that students are given extra attention during the exam period, even outside the office hours. Moreover, it is important to give timely feedback to students regarding their performance. Timely feedback helps in troubleshooting difficulties experienced by all students. Moreover, since students can have varying aptitude, it also helps in identifying students who need extra care regarding certain topics. The best way to ensure a timely feedback is to grade quizzes and assignments on time and giving detailed remarks on assignments. That allows students to perform a quick self assessment and reach out for help. 


The fourth component of {\em effective learning} is two-way learning. As students, we perceive that instructor is helping us understand the course. However, as academics, we are always learning even as instructors. Our learning can be influenced by questions raised in the classroom, evolution in the field that is being taught in the course, and feedback on the teaching style and communication. It is therefore improve to assume that even as instructors, we have the flexibility to acknowledge our knowledge gaps and the willingness to improve it by learning from students. An instructor who embraces tries to improve their knowledge is likely to inspire students to do the same. This technique, although subtle, can significantly contribute to development of students and instructors. 

The final component of my teaching philosophy is to uphold and enforce ethical values of academia. As I mentioned earlier, teaching must be considered a sanctimonious profession since instructors have the position to influence young minds. Therefore, it is important that instructors impart the moral practices to which the society adheres at large. At UCF, we are taught to have no tolerance for plagiarism, cheating, and violations of academic integrity. As an instructor, I will uphold those moral values and clearly communicate my expectation towards students regarding responsible conduct. At the same time, I earnestly believe in leading by example rather than leading by force. If students are provided all the curricular facilities solely with the intent that they learn at least some useful concepts, it will be unlikely that they resolve to unethical practices. To summarize, my teaching philosophy and style will follow the {\em effective learning} principle. My main objective will be to train students as knowledgeable and responsible individuals who can contribute to the academic circles and the society. 




\subsection*{Teaching Experiences}
In the last five years, I had the opportunity to engage in the teaching experience with students from high school to Ph.D. I have also served as a teaching assistant for two courses, namely Foundations of Computer Security and Privacy, and Natural Language Processing. Alongside, I have informally mentored the undergraduate students in their final year projects and motivated them to pursue graduate studies. The diversity of experience gained from interacting with high school students and Ph.D. students, helped me understand the requirements of constructive teaching at various levels. Below, I succinctly describe my teaching experience over the last five years.

As Masters student at LUMS, I was tasked to help the undergraduate students in their final year projects. For that purpose, I conducted weekly meetings with students and assigned them tasks. Based on the project requirements, I helped them select courses that were relevant to the project \ie Machine Learning, Data Mining, Stochastic Processes. Accordingly, I also attended those courses to stay updated with the course material and touch upon them during weekly meetings. Among the four students I assisted, three pursued their graduate studies in the USA.

At UCF, I volunteered for the outreach program organized by Camp Connect. In that activity, I taught the fundamentals of computer security to high school students. I noticed that the upcoming generation is more aware about technology. While it makes it easier to communicate with them, it also makes it challenging to stay updated with new advancements in the field. To make the best use of this situation, I structured more interactive lectures that enabled a collaborative learning experience for me and the students. Moving forward, I am noticing that research will expand from undergraduate students to high school students. NSF has a Computing K-12 Education plan that is focused on involving high school students in computer science education. I am hoping that my experience will be useful for such programs in the future. 


My teaching assistant experience was also a profound experience that motivated me to join academia. I assisted Professor David Mohaisen in Foundations of Computer Security and Privacy, and Natural Language Processing. As a TA, my responsibilities included helping the course instructor in preparing and grading course material, assignments, and exams. For each assignment, I gave a detailed feedback to each student and helped them during the weekly office hours. Following the teaching philosophy described earlier, I ensured reachability and responsiveness during the exam season and assisted them in improving their shortcomings. For the security course exams, we took a unique approach to prepare exams based on real-world parallels. The exam questions were interesting case studies each of which captured a security model taught in the class. We observed that students performed well in that exam, demonstrating good analytical skills. I will continue that approach in the courses that I will teach. In each course, we assigned course projects in the class. We encouraged students to select projects that were closely related to the course material as well as their Ph.D. To further sharpen their skills, the project report and deliverables were modeled on the research paper format (\ie IEEE style report formatting and code repositories). For students who were not familiar with the format, I provided detailed instructions and project samples as guidelines. We received good feedback from studetns in both courses. 



\vspace{2mm}
\BfPara{Teaching Interest} During my coursework and research, I have covered a wide range of topics in computer science and mathematics including computer and network security, malware analysis, design and analysis of algorithms, computer networks, operating systems, advanced computer architecture, machine learning, and advanced applied stochastic processes. If required, I can teach similar courses at undergraduate and graduate levels. I will be particularly interested in teaching courses related to security and networks since they are rapidly evolving domains with a high research potential. Based on research experience, I will revise the course content to include new concepts and techniques that can open new directions for students and help them solve cutting edge research problems. 

Additionally, I will introduce new courses related to blockchains, smart contracts, and Internet measurements. These courses have been introduced in notable universities to familiarize students with emerging concepts in computer science and give them an advantage in the job market. I am also motivated to teach special topics courses where students are taught how to read, summarize, and critique other research papers, and often apply those research techniques in other useful directions. 


\vspace{2mm}
\BfPara{Student Advising} I was fortunate to be advised by the most skillful advisors during my Masters and Ph.D. As an advisor to Masters and Ph.D. student, I will follow the same advising style that has helped my and my colleagues during our Ph.D. My own learning is that advising is the skill to train students, help them achieve their milestones, and enable them to conduct independent research. As an advisor, I will initially help my students to understand the research culture and select the topic that they want to work on. Concurrently, I will introduce them to the preliminaries of paper writing, paper reviewing, rebuttals, peer review process, and academic milestones. Once the students acclimatize with the research culture and develop preliminary skills, I will help them identify the core research problems in the areas that they are interested in. The advising style will be hands on training the first year where students will learn the basics of problem solving and effectively translating their coursework to support their research. Eventually, once the students start to publish, they will get more independent to identify and solve new problems in their domain. I will regularly meet students to help them and navigate their research. I will also ensure a healthy and interactive lab environment where students can peacefully work and engage with their colleagues. Additionally, I will ensure that my students strike a work life balance and do not work on the weekends. Finally, the another key aspect of research advising is teaching students to deal with rejections. Paper rejections can be stressful and discouraging for students, especially in the start of their Ph.D. To alleviate such disappointments, I will council them on handling rejections and improving their works through iterative feedback. 

% I am capable of teaching most introductory computer science courses, I will also be capable of teaching courses such as algorithms, computer systems (operating systems, networked systems), and computer security, both graduate and undergraduate. 


\end{document}


%Foundations of Computer Security and Privacy 
%Natural Language Processing 