\documentclass{NSF}

\usepackage{xspace}
\newcommand{\BfPara}[1]{{\noindent\textbf{#1.}}\xspace}
\newcommand{\etal}{{\em et al.}\xspace}
\newcommand{\etc}{{etc.}\xspace}
\newcommand{\ie}{{\em i.e.}\xspace}
\newcommand{\eg}{{\em e.g.,}\xspace}
\usepackage{marvosym}
\usepackage[bitstream-charter]{mathdesign}
\usepackage[T1]{fontenc}
\newcommand{\univ}{Old Dominion University}

% Let us define our framework name HERE!
\newcommand{\ours}{RFlow$^+$}
\newcommand{\ourIs}{InstaMeasure}
\newcommand{\ourFR}{FlowRegulator}
\begin{document}


% B. Project Summary
\title{Diversity Statement}\\
\name{Muhammad Saad}
\\\rule{\textwidth}{1.5pt}\vspace{3mm}
%\Mobilefone{ +1 (407) 840 8881\ \ \ \ \       }
%\Mobilefone{ +82 (10) 2052 7777}\\
%{\textbf{Email}: r.h.jang@knights.ucf.edu}
%\vspace{-8mm}
\thispagestyle{empty}

% 1) Start with a broad and general statement
% 2) Provide a personal background
% 3) Show experience 
% 4) Qualify 
In the times of globalization, embracing diversity is critical to success and progress. We are now realizing the impact of multicultural societal structures on our productivity and growth. At the same time, multicultural environments require us overcome communication barriers, accommodate underrepresented and marginalized communities, promote shared sense of common good, and collectively uphold our core values of humanism. In the current times, we often see situations where these values are challenged by unfortunate events. However, I am also proud to see our strong resolve in combating those challenges by emphasizing the need for diversity and removal of biases. To that end, diversity, equality, and inclusion are integral to my core principles and will play a central role in my career. 

My views on diversity and inclusion is shaped by personal experiences and societal observations over the years. I was born and raised in lower middle class family in Pakistan. During my studies, I moved between various cities each of which brought a cultural shock and sense of alienation. My initial schooling was in the outskirts of a small city, which was particularly challenging due to hostile environment. At the age of nine, I moved to the Dominican Convent School, where I made friends with children of different religions, backgrounds, ethnicity, and race. Dominican Convent significantly broadened by vision and perspective about gender and religious diversity by constructing our core values around respect and kindness. I completed my undergraduate at NUST, which is an institution that adopts a military inspired culture of discipline and uniformity. After that, I completed my masters at LUMS, which offers the most liberal education environment in the country. At LUMS, I actively took part in gender equality initiatives and outreach activities. I also mentored a group of outreach students by giving them preliminary course lectures on computer science. For Ph.D., I moved to the USA, where I had a broad exposure to racial, political, religious, and gender diversity. In my LAB, we have students from different countries including China, South Korea, USA, Palestine, Turkey, and India. Moreover, I have traveled to various conferences where I connected with researchers from all continents. The broad spectrum of experience helped develop personal skills that allowed me to communicate, interact, and collaborate with a diverse set of people. At the same time, I also keenly observed the societal impediments and biases that lead to divide and marginalization of communities. My learning is that embracing diversity and provisioning inclusive and safe spaces are the only way forward to prevent such divide in our generation. 


I have also demonstrated my acknowledgement for diversity by taking part in several initiatives at UCF and Orlando city. In 2019, I attended N$^{2}$Women Panel Discussion and Mentoring Session.\footnote{Co-located in ACM CoNEXT 2019; organized by my advisor David Mohaisen.} N$^{2}$Women is a discipline-specific community that fosters connections among under-represented women in computer science. I have also attended conferences and workshops on raising awareness regarding cultural diversity. Most recently, I participated in a workshop on {\em ``Community Connections - Diversity and Inclusion in Your Neighborhood''} organized by Orange County Neighborhood Services Division. During my internship at Information Sciences Institute, I met a colleague from Biomedical background who wanted to switch to computer science and work on data mining. I mentored her by providing the useful material and course contents on data science. I also helped in job search, and she is currently a Data Scientist at a startup called Twenty in San Francisco. At UCF, I also engage with colleagues at Bridges International. Each year, we welcome incoming international students and help them with housing and basic needs. We also organize outdoor activities to help them recover the from the cultural shock and express their feelings with comfort. To summarize, I try to contribute to diversity and inclusion initiates at personal level as well at the community level.


As a faculty member at \univ, I will uphold my commitment to diversity and inclusion by grounded my principles in equality and opportunity for all. I will take part in activities that remove racial, religious, gender, and cultural imbalance. Furthermore, I will ensure that my research efforts extend to underrepresented groups and maximize their participation in education and co-curricular activities. In order to build a progressive community, we must extend equal opportunities to all members in that community. This manifesto will be a central part of my future endeavors.  



\end{document}
