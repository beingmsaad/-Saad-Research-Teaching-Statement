\documentclass{NSF}

\usepackage{xspace}
\newcommand{\BfPara}[1]{{\noindent\textbf{#1.}}\xspace}
\newcommand{\etal}{{\em et al.}\xspace}
\newcommand{\etc}{{etc.}\xspace}
\newcommand{\ie}{{\em i.e.}\xspace}
\newcommand{\eg}{{\em e.g.,}\xspace}
\usepackage{marvosym}
\usepackage[bitstream-charter]{mathdesign}
\usepackage[T1]{fontenc}
\newcommand{\univ}{Old Dominion University}

% Let us define our framework name HERE!
\newcommand{\ours}{RFlow$^+$}
\newcommand{\ourIs}{InstaMeasure}
\newcommand{\ourFR}{FlowRegulator}
\begin{document}


% B. Project Summary
\title{Diversity Statement}\\
\name{Muhammad Saad}
\\\rule{\textwidth}{1.5pt}\vspace{3mm}
%\Mobilefone{ +1 (407) 840 8881\ \ \ \ \       }
%\Mobilefone{ +82 (10) 2052 7777}\\
%{\textbf{Email}: r.h.jang@knights.ucf.edu}
%\vspace{-8mm}
\thispagestyle{empty}

% 1) Start with a broad and general statement
% 2) Provide a personal background
% 3) Show experience 
% 4) Qualify 
In times of globalization, embracing diversity is critical to a progressive and successful social structure.  In a multicultural society, embracing diversity requires overcoming communication barriers, accommodating underrepresented and marginalized communities, promoting a shared sense of common good, and collectively upholding our core values of humanism. Although these values seem to be logical imperatives to be adopted by all, we must also acknowledge that we are living in challenging times of a social divide. I earnestly believe that diversity and inclusion are our only tools to combat the social divide. Therefore, they are integral to my core principles and will play a central role in my career.

\vspace{2mm}
My views on diversity and inclusion are shaped by personal experiences and societal observations over the years. I was born and raised in a lower-middle-class family in Pakistan. During my studies, I moved between various cities, each of which brought a cultural shock and a sense of alienation. My initial schooling was on the outskirts of a small city, where students were disciplined with harsh punishments. That was a particularly challenging experience since I have never considered the strict mentoring of young kids to be appropriate. However, what I learned was that tough times make life-long friendships and a sense of companionship. At the age of nine, my parents transferred me to the Dominican Convent School, where I made friends with children of different religions, backgrounds, ethnicity, and race. The Convent school environment significantly broadened my vision and perspective about gender and religious diversity by crystallizing our core values around respect and kindness. 

\vspace{2mm}
After my schooling, I completed my undergraduate at NUST, which is an institution that adopts a military-inspired culture of discipline and uniformity. Later, I completed my Masters at LUMS; an institution renowned to provide the most liberal education environment in the country. At LUMS, I actively took part in gender equality initiatives and outreach activities. I also mentored a group of outreach students by giving them preliminary course lectures on computer science. For Ph.D., I moved to the USA, where I had broad exposure to racial, political, religious, and gender diversity. In our Lab, we have students from different countries including China, South Korea, USA, Palestine, Turkey, and India. I must credit Professor Mohaisen to have enabled an interactive and productive lab environment that empowered us to freely engage with each other despite our backgrounds, religious beliefs, and ethnicity. Moreover, during my Ph.D., I have traveled to various conferences where I connected with researchers from all continents. The broad spectrum of experience allowed me to develop skills to communicate, interact, and collaborate with a diverse set of people. At the same time, I also keenly observed the societal impediments and biases that lead to the divide and marginalization of communities. My learning is that embracing diversity and provisioning safe spaces are the best tools to prevent a social divide. 

\vspace{2mm}

I have also demonstrated my understanding of diversity and inclusion by taking part in several initiatives at UCF and Orlando city. In 2019, I attended N$^{2}$Women Panel Discussion and Mentoring Session.\footnote{Co-located in ACM CoNEXT 2019; organized by my advisor David Mohaisen.} N$^{2}$Women is a discipline-specific community that fosters connections among under-represented women in computer science. I have also attended conferences and workshops on creating awareness regarding cultural diversity. Most recently, I participated in a workshop on {\em ``Community Connections - Diversity and Inclusion in Your Neighborhood''} organized by Orange County Neighborhood Services Division. At UCF, I also engage in the activities organized by Bridges International. Each year, we welcome incoming international students and help them with housing and basic needs. We also organize outdoor activities to help them recover from the cultural shock.


\vspace{2mm}
As a faculty member at \univ, I will uphold my commitment to diversity and inclusion by grounded my principles in equality and opportunity for all. I will take part in activities that remove racial, religious, gender, and cultural imbalance. Furthermore, I will ensure that my research efforts extend to underrepresented groups and maximize their participation in education and co-curricular activities. In order to build a progressive community, we must extend equal opportunities to all members in that community. This manifesto will be a central part of my future endeavors.  



\end{document}
