\documentclass{NSF}

\usepackage{xspace}
\newcommand{\BfPara}[1]{{\noindent\textbf{#1.}}\xspace}
\newcommand{\etal}{{\em et al.}\xspace}
\newcommand{\etc}{{etc.}\xspace}
\newcommand{\ie}{{\em i.e.}\xspace}
\newcommand{\eg}{{\em e.g.,}\xspace}
\usepackage{marvosym}
\begin{document}


\title{Research Statement}\\
\\\rule{\textwidth}{1.5pt}\vspace{3mm}

My research interests are focused on the security of blockchains and distributed systems. My Ph.D. work is dedicated to a systematic exploration of the blockchain attack surface actuated by the cryptographic constructs of the blockchain data structure, the information exchange in the peer-to-peer (P2P) blockchain network, and the application-specific deployment of blockchains in cryptocurrencies, smart contracts, and audit logs. My work on blockchain systems has led to the discovery of (1) DDoS attacks on their memory pools, (2) partitioning attacks on their P2P networks, (3) asynchrony in their network configuration, and (4) illicit mining of their cryptocurrency tokens through cryptojacking. Concurrently, I have also developed countermeasures for those attacks by proposing refinements to the existing consensus protocols, designing optimal network setups that enable secure and efficient information exchange, and constructing application-layer defenses to neutralize cryptojacking. All these works have been consolidated in the first systematic survey on the blockchain attack surface published in the IEEE Communication Surveys and Tutorials. 


Alongside the attack surface analysis, I have pursued two other thrusts related to the application of blockchains in cryptocurrencies and audit systems. The first thrust uses the cryptocurrency network artifacts to develop machine learning models for accurate price predictions. The price prediction methodology leveraged the strong correlation between the cryptocurrency price and user behavior that can be characterized using machine learning techniques. I deployed that approach to predict the price of Bitcoin and Ethereum and my paper won the best paper award at IEEE Systems Journal. The second thrust was on instrumenting the requirements for legacy systems where blockchain primitives can be usefully applied to achieve high security and provenance. In that space, I observed the efficacy of blockchains in the audit applications which I used to (1) harden the security of audit logs by distributing control among multiple application instances, and (2) provide audit log provenance by leveraging the append-only blockchain structure. My work on blockchain-based audit logs won the best paper award in DLoT 2018.   


In terms of novelty and impact, the most notable projects in my research are related to Bitcoin partitioning attacks which I have incorporated in my dissertation. In the following, I briefly discuss those works along with my future research plans. 

\section{Ph.D. Dissertation}
The Bitcoin network is the most dominant example of a blockchain system. The Bitcoin network is actuated by nodes that form an overlay P2P network supported by a physical network composed of Autonomous Systems (ASes). Originally, Bitcoin was conceived as a democratic network where nodes (1) had a uniform mining power, (2) the network was {\em lock-step} synchronous, and (3) nodes were geographically distributed across the physical network. Since 2009, the Bitcoin network has scaled up to more than 10K nodes while significantly departing from its ideal configurations. In the current setup, the mining power is centralized across a few nodes and those nodes have clustered among few ASes. Moreover, there is a sharp decline in the blockchain synchronization which is pertinent to ensure the blockchain {\em safety}. My Ph.D. dissertation dissects various attributes of the Bitcoin network to expose its vulnerability to partitioning attacks.





\BfPara{Partitioning Attacks} The first component of my dissertation conceptualizes spatial, temporal, spatio-temporal, and logical partitioning attacks on the Bitcoin network. In the spatial partitioning attack, I reported the increasing centralization of Bitcoin nodes and mining pools across the ASes, which increases the risk of BGP hijacks. In the temporal partitioning attacks, I observed weak network synchronization over the blockchain ledger which allows an adversarial mining pool to fork the blockchain and corrupt the view of honest nodes. In the spatio-temporal partitioning attacks, I examined the overlap between spatial and temporal attacks to expose the most vulnerable network state. The spatio-temporal attack can lead to hard forks by preventing fork recovery.  Finally, in the logical partitioning attack, I inspected the software vulnerability across dominant the Bitcoin Core deployments that put the Bitcoin users at the risk of remote shutdown and wallet theft. 

I also proposed the attack countermeasures, one of which was a {\em Routechain} that prevents BGP hijacks. The work on partitioning attacks appeared in the notable distributed systems conference (ICDCS 2019), while {\em Routechain} appeared in a major blockchain conference (ICBC 2019). My work on partitioning attacks increased the awareness about increasing network centrality across ASes. It was soon noticed by the Bitcoin developers community that developed a new technique called {\em Asmap} for Bitcoin. {\em Asmap} is a refinement to the Bitcoin addressing mechanism that creates AS-level diversity among Bitcoin nodes. 

\BfPara{HashSplit Attack} The second component of my dissertation exposes the gap between theoretical models of Bitcoin and its real-world implementation. First, I constructed the Bitcoin ideal functionality that preserves the common-prefix and chain quality properties in {\em lock-step} synchronous and {\em non lock-step} synchronous communication models. Next, I set up a large measurement apparatus to detect 359 mining nodes among the 29K non-mining nodes, and monitor the information flow among them. The results revealed a high disparity in the ideal functionality and the real-world deployment, showing that the Bitcoin network is actually asynchronous which contradict the Nakamoto's proposed model. I further demonstrated that the asynchronous network significantly amortizes the cost for the 51\% attack and violates the common-prefix and chain quality properties with a high probability. I also proposed a new attack called {\em HashSplit} that effectively splits the network hash rate by exploiting block propagation delay and forcing miners to mine on a forked chain. The work is the first attempt to conduct a data-driven study that experimentally demonstrates the truly asynchronous nature of the Bitcoin network. Additionally, through a fine-grained analysis, I showed that Bitcoin network synchronization is deteriorating over time due to block propagation delay. To improve the network health, I modified the Bitcoin Core client to closely emulate the {\em lock-step} synchronous network and resist asycnhrony. The work is currently under review in S\&P 2021. 


\BfPara{Root Cause Analysis} My prior work on temporal partitioning attacks and {\em HashSplit} assumed that weak network synchronization was purely due to the increasing network size. The assumption was inspired by prior notable works that proposed a relationship between network size and blockchain synchronization. To concretely evaluate that proposition, I used my data collection system to perform a longitudinal analysis of network synchronization from 2019 to 2020. The results refuted the proposition by showing that in two years, the network size remained stable ($\approx$10K nodes), the network synchronization deteriorated constantly. In 2019, among $\approx$10K nodes, 72\% nodes had an up-to-date blockchain, while in 2020, the network synchronization reduced to 62\%. The new results proved that network synchronization cannot be solely parameterized by the network size, and therefore mandate a root-cause analysis to investigate other factors that contribute to the blockchain synchronization. The root-cause analysis revealed that the Bitcoin block propagation is influenced by an {\em unreachable} network segment that dominates the overall network. The {\em unreachable} network has not been systematically analyzed in prior works. Moreover, I observed that the {\em unreachable} network contributes to slower block propagation and weak network connectivity. I also discovered 73 malicious Bitcoin nodes that broadcast false IP addresses to weaken the Bitcoin communication model. Moreover, I observed a high network churn in the {\em reachable} network that also deteriorated block propagation and network synchronization. Consolidating all the insights, I added new features in Bitcoin Core to strengthen the network connectivity and improve the blockchain synchronization.  

My Ph.D. dissertation has made foundational contributions to the blockchain systems community by discovering new attack vectors, identifying network anomalies, performing root-cause analysis to pinpoint bottlenecks, and proposing novel attack countermeasures and protocol refinements. During my work, I collaborated with notable researchers from the University of Florida, University of Southern California, Army Research Lab, the Inha University, and the Bitcoin Core developers community.  



\section{Future Research Plan}
My future research plans include continuation of the ongoing research on blockchain systems, expanding into the payment channel networks, and pursuing tangential research in new domains to address more pressing challenges faced by our community. 

We are moving towards the age of digital currency where digital wallets and {\em contactless} payment systems will replace paper currency and credit cards. In the realm of digital currencies, Bitcoin and Ethereum have provided blueprints that can be used to develop more sophisticated payment systems. Moving forward, we cannot solely rely on Bitcoin and Ethereum to meet the requirements of digital payment systems since they suffer from (1) high volatility, (2) low scalability, (3) high energy consumption, and (4) weak security guarantees. Realizing the limitations of existing models and the need for more scalable and secure designs, several notable companies are  investing in the blockchain and digital payment networks including Facebook, Visa, Amazon, IBM, Paypal, and JP Morgan. Each of these companies is trying to build a ``one size fits all'' blockchain system that addresses all the aforementioned issues. 

Unfortunately, there is no perfect solution that can meet all security and performance requirements. Instead, and as demonstrated in my research, there is a tradeoff between security and performance in blockchain systems. For instance, in Bitcoin, the average block time is set to 10 minutes to ensure that all the network nodes synchronize over the blockchain ledger. However, by setting block time to 10 minutes, Bitcoin sacrifices the transactions throughput by merely processing 3--7 transactions per second. My work on partitioning attacks and asynchornous network shows that even the optimistic block time of 10 minutes does not achieve good network synchronization. Intuitively, to improve the network synchronization, either the network size needs to be reduced or the block time must be increased. However, increasing block time means decreasing the transaction throughput below 3 transactions per second, and reducing the network size means limiting the network participation. When applied to large companies such as Facebook and Visa, none of the aforementioned solutions will be acceptable due to large user base and requirement for a high throughput. Moreover, as also demonstrated in my work, security is a moving target and each year, we discover new forms of attacks that violate the blockchain {\em safety}. Applied to the large financial institutions such Paypal and JP Morgan, a security lapse can be detrimental. Therefore, none of these companies have come forward with a blockchain-based product that perfectly meets all these requirements. I see this as a major research opportunity in the future that can beneficial for the blockchain community at large. In the following, I present my research plan in five key thrusts.

\BfPara{Thrust 1: Security and Feasibility Analysis of PCNs} The two most recurring research question in blockchain systems are related to increasing throughput and scalability. These questions need to addressed if blockchain network are to be commercially adopted at scale by large organizations such as Facebook and Visa. A popular approach in that direction is adopting Layer 2 solutions such as payment channel networks (PCNs). PCNs provide scalable and efficient offchain transaction exchange among users. The two popular PCNs for Bitcoin and Ethereum are the Lightening network and the Raiden network, respectively. In PCNs, nodes initiate an onchain transaction on the blockchain and lock their initial balance in a smart contract. After that, nodes can exchange multiple offline transactions in a specified time after which they settle the final balance on the blockchain. The offchain interaction reduces the overhead from the main blockchain network, thereby increasing scalability and throughput. Despite these benefits, PCNs also present challenges that are unique from the blockchain network. PCNs rely on finding an optimal route to process a transaction from the sender to the receiver. The payment route comprises of intermediary nodes that can be malicious. Malicious on-path nodes can compromise balance security, value privacy, and sender/receiver anonymity. Ensuring non-blocking and privacy preserving payment process in  an adversarial environment is a key research challenge in PCNs. Another problem with PCNs, and blockchain systems in general, is enabling cross-chain swaps that allow users to exchange assets across multiple blockchain systems and PCNs. Cross-chain swaps are similar to transferring money across different bank accounts. As the number of blockchain networks increases in the future, the need for atomic cross-chain swaps will increase accordingly. This is also an important research question in PCNs.  Finally, another problem in PCNs (similar to the blockchain systems) is that PCNs are also vulnerable to partitioning attacks and network asynchrony. Therefore, all these research questions need to be addressed so that we can leverage the high throughput and scalability benefits of PCNs without sacrificing security and privacy. 

In this thrust, I will perform two research tasks to solve these challenges. First, I will construct the ideal functionality to concretely specify the desireable properties for a secure and privacy preserving PCN. That task will require rigorous theoretical modeling and in-depth understanding of the PCN ecosystem. In the second task, I will collect data from popular PCNs and contrast their real-world behavior against the ideal functionality. The outcome of both tasks will provide a clear understanding of the current problem space and the bottlenecks. In the third task, I will address those problems and perform a feasibility analysis to show what can and cannot be achieved through PCNs as scalable alternatives for blockchains. 


\BfPara{Thrust 2: Stablecoins} Price volatility is a major problem in Bitcoin and Ethereum. To stabilize price fluctuations, major companies are coming forward with the idea of a stablecoins such as Facebook's Libra. The key concept behind stablecoins is that stakeholders will be allowed to regulate the price of the cryptocurrency tokens. A natural caveat of this model is that challenges decentralization by delegating authority to a group of stakeholders. Therefore, stablecoins only work in a permissioned blockchain model, which is known to have weaker notions of privacy and anonymity. The key research question then becomes can we achieve stronger notions of privacy, anonymity, and decentralization in stablecoins? To answer this question, I will perform the following three tasks. In the first task, I will formally analyze the privacy and anonymity benchmarks set by the permissionless blockchain systems such as Bitcoin and Zcash. In the second task, I will study the key constructs of {\em permissioned} blockchains that bind user identity to user behavior. In the third task, I will use cryptographic primitives such as non-interactive zero-knowledge proofs to develop new behavior tracking models that are independent of the user identity. In other words, we can leverage the benefits of permissioned blockchains while maintaining the privacy-preserving models of the permissionless blockchains. In my industrial collaborations, I have observed that the financial systems are motivated to adopt stablecoins in the future. Therefore, research on stablecoins is part of my future research agenda. 


\BfPara{Thrust 3: Energy Efficient Consensus Protocols} Bitcoin and Ethereum use the proof-of-work (PoW) consensus protocol to maintain a consistent blockchain ledger. Over the years, PoW has become highly energy inefficient causing environmental concern. There are two main alternatives to PoW, namely proof-of-stake (PoS) and Practical Byzantine Fault Tolerance (PBFT). PoS is an energy-efficient protocol but it causes stake accumulation among rich stakeholders. Moreover, the popular PoS-based blockchain systems do not guarantee fairness where malicious nodes are penalized for their actions. On the other hand, PBFT-based blockchains suffer from low fault tolerance and high message complexity. These two limitations prevent high scalability in PBFT-based blockchains. 


To address the challenges faced by the blockchain consensus protocols, I will pursue the following task. In the first task, I will explore techniques in which PoW can be made more energy-efficient or energy consumption can be used in more beneficial tasks such as data mining. The second task will include tailoring the auction mechanism in PoS to promote decentralization and guarantee fairness. I am currently working on a PoS model that replaces direct stake commitment with percentage stake commitment to achieve decentralization. Moreover, I am using a notion of baseline stake derived from the blockchain memory pool to penalize malicious behavior and achieve fairness. This work will be continued and further refined in the future. The final task will include evaluating new techniques to amortize message complexity in PBFT to support scalability. In my prior works, I have used a multi-layered blockchain design that shards the blockchain network into various independent layers to allow parallel transaction processing. Using that approach, I was was able to reduce the message complexity and the blockchain storage overhead. However, that design was only tested in the permissioned blockchains. In the future, I will extend that approach to the permissionless blockchain systems. 





\BfPara{Thrust 4: Improving Communication Model} In the attack surface analysis, the most dominant attack vector in blockchain systems is the overlay network and the underlying physical network. The main reason for this is that the original Bitcoin proposal did not concretely specify the semantics of the P2P topology and the communication model. The proposal simply sketched a P2P network and presented an upper bound (51\% hash rate) that preserves the blockchain {\em safety} and {\em liveness}. My Ph.D. research (HashSplit Attack) takes the opposite approach to expose the assumptions made in the original proposal. I start with the security bound (51\% hash rate) and construct a P2P system under which the security bound holds. From that, I drove the conclusion that PoW-based blockchain systems are only secure against a 51\% attack in a {\em lock-step} synchronous network. Moreover, through direct network measurements, I further show that the current Bitcoin P2P network does not follow the {\em lock-step} synchronous model. Moreover, by highlighting the discrepancies between the overlap network network and the physical network, I put forward the spatial-partitioning attack that disrupts the Bitcoin communication model. In this thrust, I will try to devise more robust mechanisms that will mitigate the discrepancies between the overlay network and the physical network to defend against the spatial-partitioning attacks. Additionally, I will address the asynchrony problem in the permissionless blockchains to improve the blockchain synchronization. The current proposal proposed in my research are only valid for the current Bitcoin network size. In the future, the network size is estimated to grow and the current schemes will be insufficient to prevent the network asynchrony. Since the P2P network is common among all blockchain systems, therefore, I will extend the current analysis to other systems and perform their risk profiling. Among those systems, I will identify the ones that closely emulate a synchronous network and the methods with which they achieve that. Through a more fine-grained analysis across different blockchain systems, I will be able to outline the most useful methods to overcome the blockchain synchronization problem. 


\BfPara{Thrust 5: Trusted Execution Environment} {\em Contactless} payments are becoming more popular in the digital currency space, especially after the COVID-19 outbreak. {\em Contactless} payments means that customers will move from using paper currency and credit cards to mobile phone payments. {\em Contactless} transactions are merely an application-specific use of any payment system such as a banking system or a blockchain system. Since blockchain systems are already using the digital payments, the major change will be observed in the banking system where users will integrate credit cards with their smart phone devices and use them as digital wallets. Digital wallets have a higher risk of theft and hacking compared to paper currency and credit cards. Currently, 86\% smartphone manufactures use the Android software stack for their devices. Android suffers from various security vulnerabilities that have been well explored and demonstrated in the research community. To mitigate the risk of digital thefts and hacks, I plan to leverage the {\em Trustzone} technology supported by all major smartphone manufacturers. The {\em Trustzone} technology provides a hardware-assisted isolation between the Rich Execution Environment (REE) such as Android and the Trusted Execution Environment (TEE) such as GlobalPlatform (GP) compliant Open-Portable TEE (OP-TEE). Typically, TEEs use security by isolation techniques to minimize interactions between REEs and allowing only limited functionalities that conform with high security primitives defined by GP. Currently, most digital wallets are stored in the insecure storage of REEs (Android). Moreover, the TEE-based software solutions such as Samsung Knox do not support over-the-air installation of other software solutions that can leverage TEE functionalities. The key research questions then become (1) how to overcome the constraints of migrating sensitive applications from REE to TEE with limited support from the equipment manufacturer, and (2) how much security improvement we get from TEE. This thrust is dedicated to envision the future of payments in the smartphones-based TEEs. 


\section{Potential of Attracting Funding}\vspace{-1mm}
The core of my research and associated thrusts are timely and useful to both academia and industry, with a high potential of attracting funding from industry, local state sponsors, and national funding agencies (\eg NSF, NRF). As mentioned earlier, top companies are actively working on blockchain systems and digital currencies. In 2020, I worked as a summer intern at Visa research. During my internship I was able to understand the ongoing trends in the industry and the future challenges that need to be addressed. My industrial collaborations will result in grants and gifts to extend the ongoing research on blockchain systems. Concurrently, I will be applying for grants and awards at Facebook with the objective to contribute the Calibra project. Similarly, I will propose the TEE-based {\em contactless} payment project to banks and financial sectors to seek sponsorship for the research. 

Blockchain systems and payment networks are highly researched areas in academia now days. In all major distributed systems and security conferences such as ICDCS and NDSS, there is a special track on the security of blockchains and payment networks. Considering the open challenges in the space there is a high potential of attracting funding from national funding agencies. In recent years, NSF and DARPA have funded several research projects on blockchain systems. My research opens new directions in security evaluation of blockchains and payments systems with a high likelihood of being sponsored by those agencies. 



\section{Other research interests} \vspace{-1mm}
Besides my research in distributed systems security, I have actively worked in other research domains including \textbf{social media analysis} and \textbf{web security} (eCrime 2019). My future research will also extend in those domains 


\BfPara{Social Media Analysis} During my Masters, I worked on discovering the collusion networks and cyborgs on Twitter that were used to create a political divide among users. To detect those networks, I performed an extensive data collection of several notable Twitter accounts and (1) monitored their daily new followers and the {\em lock-step} retweeting activities of those followers. From the data and analysis, I uncovered various Twitter campaigns that were organized by collusion network to spread political narratives through trends and retweets. Through a deeper inspection, I discovered vulnerabilities in Twitter's account creating mechanism that allowed automated creation of fake accounts while bypassing the email authorization. I notified Twitter with my new findings and those vulnerabilities have been patched.  

More recently, I used Twitter data to determine public awareness regarding COVID-19 in the most affected countries. Using a large-scale data collection system, I crawled 46K trends and 622 Million tweets from the twenty most affected countries. Next, I performed a longitudnal study on the COVID-19 trends, the volume of tweets in those trends, and the user sentiment towards COVID-19 preventive measures. The results showed that in countries with a lower spread, Twitter users actively discussed the pandemic threat and the preventive measures. The study further showed that in countries with a higher spread, users exhibited a negative sentiment towards preventive measures. 

Based on my prior research experience, I will continue to use scientific tools to address social issues by using insights from social media platforms. In particular, using my training in security, I will try to detect hate campaigns and fake news that increase polarity in our social structure, making social media platforms unsafe for users. For this research, I will conduct a multi-disciplinary research in collaboration with the faculty of social science department to broaden the scope and impact our research.  

\BfPara{Web Security}



\end{document}
