\documentclass{NSF}

\usepackage{xspace}
\newcommand{\BfPara}[1]{{\noindent\textbf{#1.}}\xspace}
\newcommand{\etal}{{\em et al.}\xspace}
\newcommand{\etc}{{etc.}\xspace}
\newcommand{\ie}{{\em i.e.}\xspace}
\newcommand{\eg}{{\em e.g.,}\xspace}
\usepackage{marvosym}
\usepackage[bitstream-charter]{mathdesign}
\usepackage[T1]{fontenc}


% Let us define our framework name HERE!
\newcommand{\ours}{RFlow$^+$}
\newcommand{\ourIs}{InstaMeasure}
\newcommand{\ourFR}{FlowRegulator}
\begin{document}


% B. Project Summary
\title{Diversity Statement}\\
\name{Muhammad Saad}
\\\rule{\textwidth}{1.5pt}\vspace{3mm}
%\Mobilefone{ +1 (407) 840 8881\ \ \ \ \       }
%\Mobilefone{ +82 (10) 2052 7777}\\
%{\textbf{Email}: r.h.jang@knights.ucf.edu}
%\vspace{-8mm}
\thispagestyle{empty}

% 1) Start with a broad and general statement
% 2) Provide a personal background
% 3) Show experience 
% 4) Qualify 
In the times of globalization, embracing diversity is critical to success and progress. We are now realizing the impact of multicultural societal structures on our productivity and growth. At the same time, multicultural environments require us overcome communication barriers, accommodate underrepresented and marginalized communities, promote shared sense of common good, and collectively uphold our core values of humanism. In the current times, we often see situations where these values are challenged by unfortunate events. However, I am also proud to see our strong resolve in combating those challenges by emphasizing the need for diversity and removal of biases. To that end, diversity, equality, and inclusion are integral to my core principles and will play a central role in my career. 

My views on diversity and inclusion is shaped by personal experiences and societal observations over the years. I was born and raised in lower middle class family in Pakistan. During my studies, I moved between various cities each of which brought a cultural shock and sense of alienation. My initial schooling was in the outskirts of a small city, which was particularly challenging due to hostile environment. At the age of nine, I moved to the Dominican Convent School, where I made friends with children of different religions, backgrounds, ethnicity, and race. Dominican Convent significantly broadened by vision and perspective about gender and religious diversity by constructing our core values around respect and kindness. I completed my undergraduate at NUST, which is an institution that adopts a military inspired culture of discipline and uniformity. After that, I completed my masters at LUMS, which offers the most liberal education environment in the country. At LUMS, I actively took part in gender equality initiatives and outreach activities. I also mentored a group of outreach students by giving them preliminary course lectures on computer science. For Ph.D., I moved to the USA, where I had a broad exposure to racial, political, religious, and gender diversity. In my LAB, we have students from different countries including China, South Korea, USA, Palestine, Turkey, and India. Moreover, I have traveled to various conferences where I connected with researchers from all continents. The broad spectrum of experience helped develop personal skills that allowed me to communicate, interact, and collaborate with a diverse set of people. At the same time, I also keenly observed the societal impediments and biases that lead to divide and marginalization of communities. My learning is that embracing diversity and provisioning inclusive and safe spaces are the only way forward to prevent such divide in our generation. 


I have also demonstrated my acknowledgement for diversity by taking part in several initiatives at UCF and Orlando city. In 2019, I volunteered for N2Women Panel Discussion and Mentoring Session\footnote{Co-located in ACM CoNEXT 2019; organized by my advisor David Mohaisen.} N$^{2}$Women is a discipline-specific community that fosters connections among under-represented women in computer science. I have also attended conferences and workshops on raising awareness regarding cultural diversity. Most recently, I participated in a workshop on {\em ``Community Connections - Diversity and Inclusion in Your Neighborhood} organized by Orange County Neighborhood Services Division. 



In times of globalization and internationalization, understating diversity is important than ever.  
As a multicultural family, my wife and I lived and studied in three different countries; China, South Korea, and the United States. For this reason, we faced various difficulties and barriers as minorities in each and everyone of those countries. Thanks to this experience, I recognize those barriers which minorities can face in pursuing higher education. I believe that the key to diversity is equal opportunity, and I will strive to ensure equal opportunity to all students.  

After attending N2Women Panel Discussion and Mentoring Session\footnote{Co-located in ACM CoNEXT 2019; where I was the volunteer, assisting my PhD advisor, who was the co-organizer)}, I learned widely about the gender imbalance in academia and the difficulties females facing to become part of the relevant research communities, in most areas of computer science, as well as approaches to reduce such imbalance through active recruitment, which I have experienced first-hand, having been part of two large groups in the US and South Korea.

As a PhD student, fortunately, I have been working with enlightened advisors during my graduate studies who take diversity very seriously and as a priority. Like my role models, Prof. DaeHun Nyang (Inha University) and Prof. Aziz Mohaisen (University of Central Florida) who always provide an equal chance to talented students without considering gender, age, nationality, race, and religion, I would like to embrace such principles towards diversity. In Prof Nyang's lab, I worked with talented colleagues  from different backgrounds, gender, age, race, and socioeconomic (ranged from 20s to 40s, and from countries such as Uzbekistan, Finland, China, and Palestine (Middle East), students included males and females, etc.) 

As a PhD student in Prof. Mohaisen's lab at UCF, I had the honor to guide a female undergrad student (African American) in a research intern program sponsored by NSF. During advising, Prof. Mohaisen actively tried to understand the difficulties of the student when pursuing higher education (as a student who had to work multiple jobs while studying) and help by providing funding opportunities to her. This experience allowed me to understand that circumstances for minority students might be the severest, and various hardships, including financial, have to be tackled for them to participate in higher education. 

My experience in working with diverse students is not limited to this mentorship experience, but includes working with a large number of students from various backgrounds, as proven by my coauthored research papers. Thanks to Prof. Mohaisen, I have been working in a diverse environment, where our laboratory member consists of American (various races), Turk, Indian, Pakistanis, Korean, Palestinian, and Saudi Arabian; with different religions. In this dynamic and rich work environment, I had the chance to learn about different cultures, religious, and backgrounds, where the mutual respect has been a key element.
These instances of experience have not only helped me understand the importance of diversity in academia but inspired me to think about how to make a social contribution in different aspects and how to be a great teacher and advisor, embracing diversity as an element.

Combing my personal experience, as my advisers have been doing, I will continuously follow their footsteps and tackle difficulties and barriers that may hinder females and other underrepresented minorities in engaging with the higher education process and research. Moreover,  I will actively provide equal opportunities for them to succeed in their academic journey and to enrich the academic process by bringing all talents on board. 

\end{document}
