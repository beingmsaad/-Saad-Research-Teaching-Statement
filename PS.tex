\documentclass{NSF}

\usepackage{xspace}
\newcommand{\BfPara}[1]{{\noindent\textbf{#1.}}\xspace}
\newcommand{\etal}{{\em et al.}\xspace}
\newcommand{\etc}{{etc.}\xspace}
\newcommand{\ie}{{\em i.e.}\xspace}
\newcommand{\eg}{{\em e.g.,}\xspace}
\usepackage{marvosym}

% Let us define our framework name HERE!
\newcommand{\ours}{RFlow$^+$}
\newcommand{\ourIs}{InstaMeasure}
\newcommand{\ourFR}{FlowRegulator}
\begin{document}

% B. Project Summary
\title{Personal Statement -- \name{Rhongho Jang}}
%\name{Rhongho Jang}
\\\rule{\textwidth}{1.5pt}\vspace{1mm}
%\Mobilefone{ +1 (407) 840 8881\ \ \ \ \       }
%\Mobilefone{ +82 (10) 2052 7777}\\
%{\textbf{Email}: r.h.jang@knights.ucf.edu}
%\vspace{-3mm}
\thispagestyle{empty}

I remember when I was young, I had said that my dream is to learn all the computer-related knowledge. At that time, everyone laughed at me. Even so, I firmly believe I can do it because I am curious and persistent in computer knowledge. 

After high school graduation, I went to Korea with my mother and enrolled in the department of computer science and engineering at INHA University. Language barriers did not stop my desire for computer knowledge, but it became my biggest motivation to learn Korean, prompting me to become a native Korean speaker. In college, I not only absorbed a vast amount of knowledge like a sponge, but also got a new insight that the focus of college learning is not knowledge itself, but the efficient way to learn new knowledge. After that, I began to self-study on my interest topics (i.e., network, security) based on coursework. After graduation, I joined professor DaeHun Nyang (Inha University), who is making considerable contributions to the security area. 

During Ms.c. study, my first research topic was rogue access point detection (i.e., wireless security), which was started from an initial idea that used intentional channel interference. I completely read IEEE 802.11 standards for figuring out feasible solutions and tested them all. Since the wireless is sensitive to noises, I had to construct all my experiments (around 3000 experiments) in the night time (i.e., 1 am~7 am). Even it was time-consuming, but it helped me to understand wireless features deeply. In this research, I defeated previous solutions and developed/implemented a channel interference-based rogue access point detection solution in a smartphone. 

During Ph.D. study, I enrolled in a dual Ph.D. program that was established between the Inha University and the University of Central Florida. This program not only provided me an opportunity to study in the United Status but also allowed me to experience research in emerging technologies, especially in the field of Cybersecurity, Privacy, Algorithm, and Networked Systems. Moreover, I had a chance to work and collaborate on several projects addressing a variety of challenges and real-world applications at two research labs, namely, Security Analytics Research Lab (SEAL) at the University of Central Florida and Information Security Research Laboratory (ISRL) at INHA University, South Korea. In 5 years, I published 7  research papers (including recently accepted papers) in top-tier conferences and high-impact journals, such as IEEE INFOCOM, IEEE ICDCS, and IEEE Transactions of Mobile Computing. Note that in all accepted or published papers, I made significant contributions and listed as the first author. Moreover, I'm continuously making my efforts to my research field (having 7 papers in submission at top venues).

To the end, I have a proven record of achievements and qualifications, including having two Ph.D. degrees, a long list of training, quality publications, and critical thinkings in my research area, which will allow me to make significant contributions to the university.





\end{document}
