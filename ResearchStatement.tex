\documentclass{NSF}

\usepackage{xspace}
\newcommand{\BfPara}[1]{{\noindent\textbf{#1.}}\xspace}
\newcommand{\etal}{{\em et al.}\xspace}
\newcommand{\etc}{{etc.}\xspace}
\newcommand{\ie}{{\em i.e.}\xspace}
\newcommand{\eg}{{\em e.g.,}\xspace}
\usepackage{marvosym}
% \usepackage{libertine}
% \usepackage{libertinust1math}
% \usepackage[T1]{fontenc}

\usepackage[bitstream-charter]{mathdesign}
\usepackage[T1]{fontenc}

\begin{document}


\title{Research Statement}\\
\\\rule{\textwidth}{1.5pt}\vspace{3mm}

My research interests are focused on the security of blockchains and distributed systems. My Ph.D. work is dedicated to the systematic exploration of the blockchain attack surface actuated by the blockchain cryptographic constructs, peer-to-peer (P2P) network, and application-specific deployments. My work on blockchain systems has led to the discovery of (1) DDoS attacks on blockchain memory pools, (2) partitioning attacks on blockchain P2P networks, (3) asynchrony in the blockchain communication model, and (4) illicit mining of blockchain-based cryptocurrencies in the web ecosystem. Concurrently, I have also developed countermeasures for those attacks by proposing refinements to the blockchain consensus protocols, designing optimal network setups that enable secure and efficient information exchange, and constructing application-layer defenses to neutralize cryptojacking attacks. All these novel projects have been succinctly summarized in the first systematic survey on the blockchain attack surface, published in the prestigious IEEE Communication Surveys and Tutorials. 


Alongside the attack surface analysis, I have pursued two other thrusts related to blockchains-based cryptocurrencies and audit systems. The first thrust uses the cryptocurrency network artifacts to develop machine learning models for accurate price prediction. My paper on Bitcoin and Ethereum price prediction won the best paper award in IEEE Systems Journal. The second thrust is towards instrumenting the requirements for legacy systems where blockchain primitives can be usefully applied to improve security and ensure provenance. Towards that goal, I designed blockchain-based audit systems to (1) harden the audit logs security by distributing control among multiple application instances, and (2) provide audit log provenance by leveraging the append-only blockchain structure. My work on blockchain-based audit logs won the best paper award in DLoT 2018.   


In terms of novelty and impact, the most notable projects in my research are related to partitioning attacks, which I have incorporated in my dissertation. In the following, I briefly discuss those works along with the future research plans.  

\section{Ph.D. Dissertation}
Bitcoin is the most representative example of a blockchain system. The Bitcoin network is actuated by nodes that form an overlay P2P network supported by a physical network composed of Autonomous Systems (ASes). Originally, Bitcoin was conceived as a democratic network in which (1) nodes had a uniform mining power, (2) the communication model was presumably {\em lock-step} synchronous, and (3) the physical network did not affect the overlay network. However, since 2009, the Bitcoin network has scaled up to more than 10K nodes while significantly departing from its ideal configuration. In the current network, the mining power is centralized among a few nodes, and those nodes are clustered in a few ASes. Moreover, the communication model is asynchronous which creates an imbalance in block propagation and network synchronization. My Ph.D. dissertation dissects these attributes of the Bitcoin network to present partitioning attacks that violate the blockchain {\em safety} properties. 



\vspace{2mm}

\BfPara{Partitioning Attacks} The first component of my dissertation conceptualizes spatial, temporal, spatio-temporal, and logical partitioning attacks on the Bitcoin network. In the spatial partitioning attack, I measure the increasing centralization of Bitcoin nodes and mining pools across the ASes, which increases the risk of BGP hijacks. In the temporal partitioning attack, I measure the weak network synchronization over the blockchain ledger which allows an adversarial mining pool to fork the blockchain and corrupt the view of honest nodes. In the spatio-temporal partitioning attacks, I examine the overlap between spatial and temporal attacks to expose the most vulnerable network states. The spatio-temporal attack combines the network centrality and block propagation delay to create hard forks and prevent fork recovery. Finally, in the logical partitioning attack, I inspected the software vulnerability across dominant the Bitcoin Core deployments that put the Bitcoin users at the risk of remote shutdown and wallet theft. 

To counter partitioning attacks, I developed various countermeasures, one of which was a {\em Routechain} that prevents BGP hijacks. The work on partitioning attacks appeared in the notable distributed systems conference (ICDCS 2019), while {\em Routechain} appeared in a major blockchain conference (ICBC 2019). The partitioning attack project informed the research and development community regarding the developing network centrality across ASes. In response, Bitcoin Core developers introduced application-layer defense technique called {\em Asmap}. {\em Asmap} is a refinement to the Bitcoin addressing mechanism that creates AS-level diversity among Bitcoin nodes. 

\vspace{2mm}
\BfPara{HashSplit Attack} The second component of my dissertation exposes the gap between theoretical models of Bitcoin and its real-world implementation. In this work, first, I formulated the Bitcoin ideal functionality to specify the {\em lock-step} network communication model that preserves the common-prefix and chain quality properties. I then deployed a large-scale data collection and network measurement apparatus to (1) detect 359 mining nodes among the 29K non-mining nodes, (2) monitor block propagation and network synchronization, and (3) identify the real world communication model. The experiments revealed a high disparity in the ideal functionality and the real-world deployment, showing that the real-world Bitcoin network is asynchronous, contradicting the Nakamoto's proposed model. I formulated the discrepancy in theoretical models and real-world deployment in the {\em HashSplit} attack which exploits network asynchrony to violate the blockchain {\em safety} properties with a high probability. The{\em HashSplit} attack achieves that by leveraging natural block propagation delay to split the network hash rate and enable concurrent mining on a forked chain. The {\em HashSplit} attack is the first effort that experimentally demonstrates the truly asynchronous anatomy of the Bitcoin network. To counter the {\em HashSplit} attack, I modified the Bitcoin Core client to closely emulate the {\em lock-step} synchronous network and resist asycnhrony. The work is currently under review in S\&P 2021. 


\vspace{2mm}
\BfPara{Root Cause Analysis} In the temporal partitioning attack and the {\em HashSplit} attack, I assumed that weak network synchronization was solely due to block propagation delay that increases with the network size. The assumption was inspired by prior notable works that proposed a relationship between network size and blockchain synchronization. To evaluate that proposition, I used my data collection and measurement apparatus to experimentally validate the relationship between the network size and blockchain synchronization. I collected the Bitcoin network data from 2019 to 2020. The results showed no relationship between the network size and the blockchain synchronization, contradicting the hypothesis in prior works. In 2019 the network size was $\approx$10K nodes and the percentage of synchronized nodes was $\approx$72\%. In 2020, while the network size remained the same, the network synchronization reduced to 62\%. The results showed that the network synchronization cannot be solely paramterirzed by the network size, and therefore mandates a root cause analysis to investigate other network features that play a role in network synchronization. 

 
For the root cause analysis, I explored various aspects of the Bitcoin network that have not be been thoroughly investigated in the prior research. More specifically, I looked at the impact on network synchronization by (1) {\em unreachable} Bitcoin nodes, (2) Bitcoin addressing mechanism and message handling protocols, and (3) the Bitcoin network churn. The root cause analysis revealed that the Bitcoin block propagation is significantly influenced by an {\em unreachable} nodes that dominate the network. The Bitcoin addressing protocol does not distinguish between {\em reachable} and {\em unreachable} nodes which accounts for 88\% failures in outbound connection attempts, and as a result, weak network connectivity. I also discovered 73 malicious Bitcoin nodes that exploit the weaknesses in addressing protocol to broadcast false IP addresses and weaken the network connectivity. Moreover, I observed a high network churn in the {\em reachable} network that contributes the most to poor network synchronization. Each day $\approx$8\% nodes leave the network, replaced by new nodes that impede network synchronization. Consolidating all my insights, I added new features in Bitcoin Core to strengthen the network connectivity and improve the blockchain synchronization. The new Bitcoin Core version addresses all the notable network vulnerabilities exposed in my Ph.D. research to harden the Bitcoin security.  

\vspace{2mm}
My Ph.D. dissertation makes foundational contributions to the distributed systems security research by identifying network anomalies, discovering new attack vectors, and proposing attack countermeasures and protocol refinements. During my Ph.D., I have collaborated with notable researchers from the University of Florida, University of Southern California, Army Research Lab, the Inha University, and the Bitcoin Core developers community.  



\section{Future Research Plan}
My future research plans include (1) continuing the ongoing research on blockchain systems, (2) expanding the research to similar domains such as digital payment channel networks, and (3) initiating research in new directions tangential to the current work. In the following, I briefly describe my observations regarding the current trends in blockchains and payment channel ecosystem, followed by my research plan summarized in five thrusts.  

\vspace{1mm}
We are moving towards the age of digital currency in which digital wallets and {\em contactless} payment systems will replace paper currency and credit cards. In the last decade, Bitcoin and Ethereum have provided blueprints for a digital payment system that can be further sophisticated to meet the growing demand. Currently, Bitcoin and Ethereum cannot meet our future requirements since they suffer from (1) high volatility, (2) low scalability, (3) high energy consumption, and (4) weak security guarantees. Realizing these limitations, several notable companies are investing in the blockchain and digital payment networks, including Facebook, Visa, Amazon, IBM, Paypal, and JP Morgan. Unfortunately, there is no ``one size fits all'' model that perfectly meets all security and performance requirements. Instead, and as demonstrated in my research, there is a tradeoff between security and performance in blockchain systems. For instance, in Bitcoin, the average block time is set to 10 minutes to ensure that all the network nodes synchronize over the blockchain ledger. However, by setting block time to 10 minutes, Bitcoin sacrifices the transactions throughput by merely processing 3--7 transactions per second. My work on partitioning attacks and asynchornous network shows that even the optimistic block time of 10 minutes does not achieve good network synchronization. Intuitively, to improve the network synchronization, either the network size needs to be reduced or the block time must be increased. However, increasing block time means decreasing the transaction throughput below 3 transactions per second, and reducing the network size means limiting the network participation. When applied to large companies such as Facebook and Visa, the aforementioned solutions will be not acceptable due to large user base and requirement for a high throughput. Moreover, my research also shows that security is a moving target. Each year, we discover new attacks that can violate the blockchain {\em safety} properties. This demonstrates that the blockchain security has not fully matured, requiring more research. In large financial institutions such Paypal and JP Morgan, such weak security models cannot be acceptable for payment systems. Given several research gaps in this space and the widespread interest, I plan to pursue my future research in blockchains and payment channels using the following five thrusts. 

\vspace{2mm}

\BfPara{Thrust 1: Security and Feasibility Analysis of PCNs} The most recurring research question in blockchain systems is about achieving high scalability and throughput. This concern needs to be addressed if blockchain systems are to be commercially adopted at scale. One possible solution to achieve high scalability and throughput is using Layer 2 solutions such as payment channel networks (PCNs). PCNs provide scalable and efficient offchain transaction exchange among users. In PCNs, users initiate an onchain transaction on the blockchain and lock their initial balance in a smart contract. After the initial commitment, users can exchange multiple offline transactions in a specified time after which they settle the final balance on the blockchain. As a result, multiple offchain interactions are represented by only two blockchain transactions which reduces the communication overhead and increases scalability and throughput. 


Despite these benefits, PCNs also present challenges that are unique from the blockchain network. PCNs rely on finding an optimal route to process transaction from a sender to a receiver. The payment route comprises of intermediary nodes that can be malicious. Malicious on-path nodes can compromise balance security, value privacy, and sender/receiver anonymity. In PCNs, ensuring a privacy preserving payment process is a key research challenge. Moreover, Another problem in PCNs --and blockchain systems in general-- is enabling cross-chain swaps that allow users to exchange assets across multiple blockchain systems and PCNs. Cross-chain swaps are similar to transferring money across different bank accounts. As the number of blockchain networks increases in the future, the need for atomic cross-chain swaps will increase accordingly. Currently, we do not have a privacy-preserving payment network that support multiple cross-chain swaps in blockchains and PCNs. Therefore, this is another research question that needs to be addressed. Finally, PCNs are also vulnerable to partitioning attacks and network asynchrony. PCNs also form an overlay P2P network which is supported by a physical network of ASes. PCN users have IP addresses which can be mapped onto ASes to launch the spatial partitioning attack. 


This thrust involves solving PCN challenges. In the first task, I will construct the ideal PCN functionality to concretely specify the desirable properties for a secure and privacy preserving PCN. That task will require rigorous theoretical modeling and in-depth understanding of the PCN ecosystem. In the second task, I will collect data from popular PCNs and contrast their real-world behavior against the ideal functionality. The outcome of both tasks will provide a clear understanding of the current problem space and the bottlenecks. In the third task, I will address those problems and perform a feasibility analysis to show what can and cannot be achieved through PCNs as scalable alternatives for blockchains. 

\vspace{2mm}
\BfPara{Thrust 2: Exploring Stablecoins} Price volatility is a major problem in Bitcoin and Ethereum. To stabilize price fluctuations, major companies are coming forward with the idea of a stablecoins such as Facebook's Libra. The key concept behind stablecoins is that stakeholders will be allowed to regulate the price of the cryptocurrency tokens. A natural caveat of this model is that it challenges decentralization by delegating authority to a group of stakeholders. Therefore, stablecoins only work in a permissioned blockchain model, which is known to have weaker notions of privacy and anonymity. The key research question then becomes can we achieve stronger notions of privacy, anonymity, and decentralization in stablecoins? To answer this question, in this thrust, I will perform the following three tasks. In the first task, I will formally analyze the privacy and anonymity benchmarks set by the permissionless blockchain systems such as Bitcoin and Zcash. In the second task, I will study the key constructs of {\em permissioned} blockchains that bind user identity to user behavior. In the third task, I will use cryptographic primitives such as non-interactive zero-knowledge proofs to develop new behavior tracking models that are independent of the user identity. In other words, we can leverage the benefits of permissioned blockchains while maintaining the privacy-preserving models of the permissionless blockchains. In my industrial collaborations, I have observed that the financial systems are motivated to adopt stablecoins in the future. Therefore, research on stablecoins is part of my future research agenda. 


\vspace{2mm}
\BfPara{Thrust 3: Energy Efficient Consensus Protocols} Bitcoin and Ethereum use the proof-of-work (PoW) consensus protocol to maintain a consistent blockchain ledger. In the last eight years, PoW has become highly energy inefficient causing environmental concern. There are two main alternatives to PoW, namely proof-of-stake (PoS) and Practical Byzantine Fault Tolerance (PBFT). PoS is an energy-efficient protocol but it causes stake accumulation among rich stakeholders. Moreover, the popular PoS-based blockchain systems do not guarantee fairness where malicious nodes can be penalized for prematurely aborting the protocol. On the other hand, PBFT-based blockchains suffer from low fault tolerance and high message complexity. These two limitations prevent high scalability in PBFT-based blockchains. 


To address the blockchain consensus challenges, in this thrust, I will conduct the following tasks. In the first task, I will explore techniques in which PoW can be made more energy-efficient or the energy consumption can be leveraged for useful tasks such as data mining. There are preliminary works in this direction including Proof-of-Useful-Work protocol. I will try to extend that work by exploring new usecases for that protocol application and identifying methods of applying it in Bitcoin and Ethereum. In the second task, I will improve the auction mechanism in PoS to achieve decentralization and fairness. I am currently working on a PoS model that replaces direct stake commitment with percentage stake commitment to achieve decentralization. Moreover, I am using a notion of baseline stake derived from the blockchain memory pool to penalize malicious behavior and achieve fairness. This work will be continued and further refined in the future. The final task will include evaluating new techniques to amortize message complexity in PBFT to support scalability. In my work on blockchain-based audit logs, I have used a multi-layered blockchain design that shards the blockchain network into various independent layers to enable parallel transaction processing. Using that approach, I was was able to reduce the message complexity and the blockchain storage overhead. However, that design was only tested in the permissioned blockchains. In the future, I will extend that approach to the permissionless blockchain systems. 


\vspace{2mm}


\BfPara{Thrust 4: Improving Network Communication Model} In the attack surface analysis, I observed that the most vulnerable component of a blockchain system is the P2P network. This is due to the fact that the original Bitcoin proposal by Nakamoto did not concretely specify the network topology or the communication model. The proposal simply sketched a P2P network and presented an upper bound (51\% hash rate) that prevents double-spending. In his construction, Nakamoto made a few network assumptions that I uncovered in the the {\em HashSplit} attack analysis. I did that by taking the opposite approach (\ie first specifying the security bound and constructing the network model in which the bound holds). My analysis shows that Nakamoto assumed a completely connected P2P network that mines block in a {\em lock-step} synchronous model. However, through large-scale measurements of the real-world Bitcoin network, I show that the current Bitcoin P2P network does not follow the {\em lock-step} synchronous model. Moreover, Nakamoto did not distinguish between the Bitcoin overlay network and the underlying physical network. In the spatial partitioning attack, I demonstrate that the physical network can significantly influence the Bitcoin communication model. In the current network, Bitcoin nodes are clustered across ASes which increases the risk of BGP attacks.

This thrust will be directed towards improving the communication model in blockchain systems by executing a series of tasks. In the first task, I will continue to explore other possible assumptions made by Nakamoto regarding the P2P network. A complete understanding of the hidden network intricacies will enable us to comprehensively estimate the thread landscape. In the second task, I will develop robust countermeasures for the blockchain partitioning attacks. A major effort in that direction will be dedicated to the addressing the asynchrony problem in the blockchain systems. As the blockchain systems grow in the future, the network asynchony will increase, since a completely connected topology among $\approx$10K nodes is infeasible in practice. 

Since the P2P network is common among all blockchain systems, therefore, I will extend my work from Bitcoin to other blockchain systems and identify the ones that closely emulate a synchronous network. Those systems can aid us in devising techniques to counter partitioning attacks. 

\vspace{2mm}

\BfPara{Thrust 5: Trusted Execution Environment} {\em Contactless} payments are becoming increasing popular in the digital currency space, especially after the COVID-19 pandemic. {\em Contactless} payments replace paper currency and credit cards with smartphone payment applications that use Near Field Communication (NFC). Currently, 86\% smartphone manufactures use the Android software stack for their devices. The Android software stack suffers from various security vulnerabilities that have been actively researched in the community. These vulnerabilities can lead to online fraud, ransomware, and theft. To prevent such malicious activities, I plan to leverage the {\em Trustzone} technology that is supported by all major smartphone manufacturers. The {\em Trustzone} technology provides a hardware-assisted isolation between the Rich Execution Environment (REE) such as Android and the Trusted Execution Environment (TEE) such as GlobalPlatform (GP) compliant Open-Portable TEE (OP-TEE). Typically, TEEs use security by isolation techniques to minimize interactions between REEs and allowing only limited functionalities that conform with high security primitives defined by GP. Currently, most digital wallets are stored in the insecure storage of REEs (Android). Moreover, the TEE-based software solutions such as Samsung Knox do not support over-the-air installation of other software solutions that can leverage TEE functionalities. The key research questions then become (1) how to overcome the constraints of migrating sensitive applications from REE to TEE with limited support from the equipment manufacturers, and (2) how much security improvement we get from TEE. This thrust is dedicated to envision the future of payments in the smartphones-based TEEs. I plan to perform an attack surface analysis of smartphone devices using open-source software stacks. Next, I will conduct a feasibility analysis to enumerate the steps required to migrate REE applications to the TEE. Finally, I will conduct penetration testing of TEE applications to evaluate their security guarantees. Apart from the financial applications, the analysis will also be extended to the cryptocurrency wallets that are currently stored in the insecure REEs. 


\section{Potential of Attracting Funding}\vspace{-1mm}
The core of my research and associated thrusts are timely and useful to both academia and industry. Therefore, my future research has a high potential of attracting funding from industry, local state sponsors, and national funding agencies (\eg NSF, NRF). As mentioned earlier, renowned companies including Facebook, Visa, JP Morgan, and Amazon are actively working on blockchain systems and digital currencies. In 2020, I worked as a summer intern at Visa research where I came to know about the ongoing trends in the industry and the challenges to expect in the future. At Visa, I developed industrial collaborations which will be useful for the future sponsorship. Concurrently, I will be applying for grants and awards at Facebook with the objective to contribute to the Calibra and Novi projects. Similarly, I will propose the TEE-based {\em contactless} payment project to banks and financial institutions to seek sponsorship for the research. 

Blockchain systems and payment networks are actively researched areas in academia. All major distributed systems and security conferences such as ICDCS and NDSS, include a special track on the security of blockchains and payment networks. Considering the open challenges in the space, there is a high potential of attracting funding from national funding agencies. In recent years, NSF and DARPA have funded several research projects on blockchain systems. During my Ph.D., I have closely worked with my advisor on writing four grant proposals. Therefore, I have a preliminary training in writing award-winning proposals. This year, I will participate the NSF Networking Technology and Systems Early-Career Investigators (NeTS-ECI) Workshop that offers training and support for new faculty researchers. I will effectively use that training to attract funding for my research. 



\section{Other research interests} \vspace{-1mm}
Besides my research in distributed systems security, I have actively worked in other research domains including \textbf{social media analysis} and \textbf{web security} (eCrime 2019). My future research will also extend in those domains 


\BfPara{Social Media Analysis} During my Masters, I worked on discovering the collusion networks and cyborgs on Twitter that were used to create a political divide among users. To detect those networks, I performed an extensive data collection of several notable Twitter accounts and (1) monitored their daily new followers and the {\em lock-step} retweeting activities of those followers. From the data and analysis, I uncovered various Twitter campaigns that were organized by collusion network to spread political narratives through trends and retweets. Through a deeper inspection, I discovered vulnerabilities in Twitter's account creating mechanism that allowed automated creation of fake accounts while bypassing the email authorization. I notified Twitter with my new findings and those vulnerabilities have been patched.  

More recently, I used Twitter data to determine public awareness regarding COVID-19 in the most affected countries. Using a large-scale data collection system, I crawled 46K trends and 622 Million tweets from the twenty most affected countries. Next, I performed a longitudnal study on the COVID-19 trends, the volume of tweets in those trends, and the user sentiment towards COVID-19 preventive measures. The results showed that in countries with a lower spread, Twitter users actively discussed the pandemic threat and the preventive measures. The study further showed that in countries with a higher spread, users exhibited a negative sentiment towards preventive measures. 

Based on my prior research experience, I will continue to use scientific tools to address social issues by using insights from social media platforms. In particular, using my training in security, I will try to detect hate campaigns and fake news that increase polarity in our social structure, making social media platforms unsafe for users. For this research, I will conduct a multi-disciplinary research in collaboration with the faculty of social science department to broaden the scope and impact our research.  

\BfPara{Web Security and Privacy} In 2018, I worked on the static, dynamic, and economic analysis of in-browser cryptojacking. They main idea was to understand the {\em modus operandi} of cryptojacking websites from a browser standpoint. During that project, I learned new aspects about web ecosystem including web security, web privacy, online advertising, content monetizing and browser instrumentation. In the future, I see the development of a blockchain-based content monetizing schemes that arbitrate the interactions between users, advertisers, and publishers. The brave browser has already released the ``Basic Attention Token'' (BAT) on Ethereum blockchain that incorporates a blockchain system in the online advertisement ecosystem. However, the blockchain-based digital advertising has security and privacy risks. A merger between blockchains and web ecosystem will reveal a new attack surface that will challenge the security and privacy designs of both systems. With a sufficient experience in both domains, I will explore the security and privacy risks of integrating blockchains with digital advertising. For this effort, I will collaborate with my colleagues and collaborators in academia to jointly lead the effort on privacy-preserving online advertisement models using blockchains. 



\end{document}
