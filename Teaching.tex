\documentclass{NSF}

\usepackage{xspace}
\newcommand{\BfPara}[1]{{\noindent\textbf{#1.}}\xspace}
\newcommand{\etal}{{\em et al.}\xspace}
\newcommand{\etc}{{etc.}\xspace}
\newcommand{\ie}{{\em i.e.}\xspace}
\newcommand{\eg}{{\em e.g.,}\xspace}
\usepackage{marvosym}

% Let us define our framework name HERE!
\newcommand{\ours}{RFlow$^+$}
\newcommand{\ourIs}{InstaMeasure}
\newcommand{\ourFR}{FlowRegulator}
\begin{document}

% B. Project Summary
\title{Teaching Statement -- \name{Rhongho Jang}}
%\name{Rhongho Jang}
%\\\rule{\textwidth}{1.5pt}\vspace{1mm}
%\Mobilefone{ +1 (407) 840 8881\ \ \ \ \       }
%\Mobilefone{ +82 (10) 2052 7777}\\
%{\textbf{Email}: r.h.jang@knights.ucf.edu}
%\vspace{-3mm}
\thispagestyle{empty}
As a Ph.D. candidate, I experienced most computer science classes. However, I found that it is hard to remember all the details, and what stood out in my experience was the way of professors' delivery of the material. Thus, my goal is to teach students how to learn efficiently instead of focusing on teaching knowledge itself. A key element is to guide them to have the right values and behaviors towards their academic pursuit.

\subsection*{Teaching Philosophy and Style}
I consider a teacher as a service/content provider for the undergrad student and role model for graduate students. This thinking is coming from my personal experience of close to a decade of study as both undergrad and graduate students. As a teacher to undergraduate students, I will consider myself a content provider who provides useful content to customers (students). Feedback about the content will be essential in improving quality. Of course, it is impossible to cover all angles of knowledge in a single course, thus determining what is essential based on the current trend in both academia and industry is essential, namely, in designing undergraduate material, I am to answer: 1) why is it important? 2) where it is/can be used, and 3) is it state-of-the-art knowledge? Instead of boring teaching and practice, I will focus on answering these three questions before going through every point to achieve the active learning of students. To this end, I firmly believe that delivering knowledge is important. Yet, more important is to teach students to understand the value of knowledge and methods to learn efficiently. As an advisor of graduate students, I will strive to be a good role model for my students. To be a good advisor, I will lead by example by showing my effort and passion in research by conducting in-depth research and publishing significant papers myself, and in the process teach my students principles of research. I believe the most important thing to show is my ability in research, demonstrating both quality and ethics. 

\vspace{-3mm}
\subsection*{Teaching Experiences}
Over the past five years, I had had various opportunities to practice my teaching philosophy. During my graduate studies, I had the unique experience of being a teaching assistant for multiple courses, such as algorithms, computer security, and coding competition (the equivalent of ICPC). In the coding competition course, my responsibilities were to define the algorithms problems (with the lecturer), provide test cases (input/output datasets), and sample solutions. My primary routine was to maintain the code evaluation servers for hundreds of students (4 parallels classes). At that time, I had a chance to look into most codes, which helped me understand in-depth  1) which part of knowledge students usually cannot understand well, 2) which part of coding skills students lacked, and 3) what kind of mistakes students made. Based on this experience, I provided feedback to students for guiding them to improve their coding skills. This process not only helped me understand the challenges in teaching but also allowed me to think more about the way of delivering knowledge. As a graduate student, I was also the lead mentor of three master's students during my Ph.D. (and under the guidance of my advisor). I had an opportunity to experience guiding students to conduct research and experiments scientifically. Even though delivering advanced knowledge and training material was challenging, it was a rewarding experience possible only through persistence. This unique training also provided me with the skills required to succeed as a mentor: when under the pressure of work, one may overlook the health of the environment in which students work. To ensure a healthy work environment, I always had time allocated to hear from my mentees about their concerns with their studies outside of the lab. Needless to say, having that little time allocated to understand students better had the most significant impact on their productivity. In the future, I will keep that as a principle in my teaching and mentorship. 

\BfPara{Teaching Interest} While I will be capable of teaching most introductory computer science courses, I will also be capable of teaching courses such as algorithms, computer systems (operating systems, networked systems), and computer security, both graduate and undergraduate. 

%Whenever they felt down and stressful, I listened to their difficulties and shared my experience and method to face setbacks. With their graduation, I deeply feel the significance and value of education, and I am still thankful for this wonderful experience.
%Endless, I would like to practice more about my teaching goals by improving myself in knowledge delivering, mentoring, and advising. Eventually can have my distinguishable teaching philosophy to help more talented students.




\end{document}
